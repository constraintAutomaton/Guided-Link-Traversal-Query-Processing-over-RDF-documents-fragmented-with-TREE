\section{Conclusion}

The publication of linked data in SPARQL endpoints is not always a sustainable approach due to unavailability and cost problems.
Our work is centered around decentralized alternatives for linked data publication.
Our preliminary results show that our rule-based reachability criterion can significantly reduce the execution time of queries aligned with hypermedia description constraints compared to predicate-based reachability
opening the possibility for faster and more versatile traversal-based query execution over fragmented RDF documents.
Our experiment also highlights that the size of the internal data store might have more impact on performance than noted in previous studies.
In future work, we will perform more exhaustive evaluations of other types of domain-oriented fragmentation strategies,
and investigate how to generalize our approach to support more expressive online reasoning for online source selection during traversal queries.

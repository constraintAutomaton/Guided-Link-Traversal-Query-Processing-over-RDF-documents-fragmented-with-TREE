\section{Introduction}

\href{https://lod-cloud.net/#diagram}{The increasing amount of available Linked Data on the Web},
prompts the need for efficient query interfaces, with SPARQL endpoints as the most prominent one.
During a typical SPARQL query execution, the interface takes the whole query load and delivers the results to the client.
This may cause high workloads and is partially the reason for the historically low
availability levels~\cite{aranda2013}.
Academics have made efforts to introduce alternative Linked Data publication methods
that enable clients participating in the query execution process.
The goal is to lower server-side workloads while keeping fast query execution to the client~\cite{Azzam2021}.
The TREE hypermedia specification is an effort in that direction~\cite{ColpaertMaterializedTREE, lancker2021LDS},
that introduces the concept of domain-related fragmentation of large RDF datasets.
For example, in the case of periodic measurements of sensor data,
the fragmentation can be made based on the publication date of each data entity.
TREE aims to fragment datasets in a way that enables clients to easily fetch subsets that fully answer a given query.
The data inside a fragment are bounded with constraints that are expressed using hypermedia descriptions~\cite{thomasFieldingPhdThesis}.
More precisely, each fragment declaratively describes the constraints of the data it contains, and links to other fragments.
Since TREE fragments are hyperlinked Linked Data documents,
clients must traverse over these documents to find data,
which makes Link Traversal Query Processing (LTQP)~\cite{Hartig2016} a suitable technique for answering SPARQL queries over it.

LTQP typically starts with a set of seed URLs that are dereferenced.
From these dereferenced documents, links to other documents are dereferenced recursively.
So far, applications on top of TREE datasets~\cite{ColpaertMaterializedTREE, lancker2021LDS}
have resorted to custom traversal implementations to find data matching specific data needs.
Our aim in this work is to explore how to execute generic SPARQL queries over TREE datasets through LTQP.
We do not aim to extend the existing TREE specification,
but merely to introduce LTQP-specific link pruning techniques that exploit the structural properties of TREE,
similar to the exploitation of structural properties when performing LTQP over the Solid environment\cite{taelman2023}.
Specifically, we will make use of the SPARQL FILTER expression to prune links that match the constraints of TREE fragments.

As a running example throughout this paper, we consider the publication of sensor data.
For example, the query the listing \ref{lst:system} targets the DAHCC~\cite{dahcc_resource} dataset.
We have created queries to get the measures between a specific time interval (the filter expression will vary in our experiment) 
and information about the sensor using a metadata file, which is available at listing \ref{lst:system}.

\begin{figure}[h]
    \begin{minipage}{0.50\textwidth}
        \centering
        \lstinputlisting[language=,frame=single]{code/example_sparql_query.ttl}
    \end{minipage}
    \hspace{0.05\textwidth}
    \begin{minipage}{0.40\textwidth}
        \centering
        \lstinputlisting[language=,frame=single]{code/example_tree_relation.ttl}
    \end{minipage}
    \caption{On the left, is a SPARQL query to get sensor measurements and metadata of a fragmented dataset.
    On the right, is the hypermedia description of the fragment following the TREE specification, the next fragment
    has a constraint on publication time $?t>= \text{2022-01-03T09:47:59.000000}$.}
        \label{lst:system}
\end{figure}
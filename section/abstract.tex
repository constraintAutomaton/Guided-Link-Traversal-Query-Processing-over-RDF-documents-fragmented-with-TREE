\begin{abstract}
% <!-- Context -->
Link Traversal queries on the web face challenges in source selections and completeness definition due to the size of the web.
Reachability criteria define completeness and source selection policies based on the traversal of links in the internal data store of query engines.
However, the size of the search domain remains the bottleneck of the approach.
% <!-- need -->
Web environments often have structures exploitable by query engines to perform more precise source selection.
Current criteria rely on using information from the query definition and regular expressions to select sources.
Thus, it is difficult to use them to traverse environments where logical expressions indicate the location of resources.
For example, the traversal of documents respecting expressions such as "the data produced after the first of September are stored at \texttt{ex:afterFirstSeptember}" are difficult to express using current reachability criteria.
% <!-- task -->
We propose to use a rule-based reachability criterion that captures logical statements expressed in hypermedia descriptions within linked data documents to define the source selection policy of traversal queries.
%<!-- object -->
In this poster paper, we show how the Comunica Web querying engine is modified to, during link traversal, take hints from the TREE hypermedia controls, to more efficiently query over subsets of a sensor dataset published using the TREE hypermedia specification.
% <!-- findings -->
Our preliminary findings show that using this strategy, the query engine can significantly reduce the number of HTTP requests and the query execution time without sacrificing the completeness of results when executing queries aligned with the hypermedia controls.
% <!-- conclusion -->
Given the promising result of this initial approach, we will extend our implementation beyond a time-based criterion, to support general reasoning over linked data in the source selection process of traversal queries.
\end{abstract}

%% The first command in your LaTeX source must be the \documentclass command.
%%
%% Options:
%% twocolumn : Two column layout.
%% hf: enable header and footer.
\documentclass[
% twocolumn,
% hf,
]{ceurart}

%%
%% One can fix some overfulls
\sloppy

%%
%% Minted listings support 
%% Need pygment <http://pygments.org/> <http://pypi.python.org/pypi/Pygments>
\usepackage{listings}
%% auto break lines
\lstset{breaklines=true}

\usepackage{svg}
\usepackage{graphicx}
\usepackage{sepfootnotes}
\usepackage{csquotes}
\usepackage{subfig}
\usepackage{listings}
\usepackage{xcolor}
\usepackage{hhline}

% Packages
\usepackage[dvipsnames,svgnames]{xcolor} % text color
\usepackage[normalem]{ulem} % wavy underlines

% Comments
\newcommand{\todo}[1]{\noindent\textcolor{red}{{\bf \{TODO}: #1{\bf \}}}}
\newcommand{\TODO}[1]{\todo{#1}}
\newcommand{\citeneeded}{\textcolor{red}{{\bf [?!]}}}
\newenvironment{draft}{\color{gray}}{\color{black}}

% Reviewers
\newcommand{\rt}[1]{\noindent\textcolor{red}{{\bf \{RT}: #1{\bf \}}}}
\newcommand\rv[1]{{\color{RubineRed}\textbf{RV}: #1}}
\newcommand\jr[1]{{\color{Purple}\textbf{JR}: #1}}

% Annotations
\makeatletter
\font\uwavefont=lasyb10 scaled 700
\def\spelling{\bgroup\markoverwith{\lower3.5\p@\hbox{\uwavefont\textcolor{Red}{\char58}}}\ULon}
\def\grammar{\bgroup\markoverwith{\lower3.5\p@\hbox{\uwavefont\textcolor{LimeGreen}{\char58}}}\ULon}
\def\phrasing{\bgroup\markoverwith{\lower3.5\p@\hbox{\uwavefont\textcolor{RoyalBlue}{\char58}}}\ULon}
\let\rephrase\phrasing
\newcommand\remove{\bgroup\markoverwith{\textcolor{red}{\rule[0.5ex]{2pt}{0.4pt}}}\ULon}
\newcommand\insertion{\bgroup\markoverwith{\textcolor{Green}{\rule[-0.5ex]{2pt}{0.6pt}}}\ULon}
\makeatother


\lstset{
    basicstyle=\ttfamily\footnotesize,
    backgroundcolor=\color{gray!10},
    numbers=left,
    numberstyle=\tiny\color{gray},
    stepnumber=1,
    numbersep=5pt,
    showstringspaces=false,
    breaklines=true,
    frame=single
}

%%
%% end of the preamble, start of the body of the document source.
\begin{document}

%%
%% Rights management information.
%% CC-BY is default license.
\copyrightyear{2024}
\copyrightclause{Copyright for this paper by its authors.
  Use permitted under Creative Commons License Attribution 4.0
  International (CC BY 4.0).}

%%
%% This command is for the conference information
\conference{The ISWC 2024 Posters and Demos Track}

%%
%% The "title" command
\title{Exploring a rule-based approach for source selection in link traversal queries of sensor data}


%%
%% The "author" command and its associated commands are used to define
%% the authors and their affiliations.
\author[1]{Bryan-Elliott Tam}[%
email=bryanelliott.tam@ugent.be,
orcid=0000-0003-3467-9755
]
\cormark[1]

\author[1]{Ruben Taelman}[%
orcid=0000-0001-5118-256X,
email=ruben.taelman@ugent.be,
url=https://www.rubensworks.net,
]
\author[1]{Pieter Colpaert}[%
orcid=0000-0001-6917-2167,
email=pieter.colpaert@ugent.be,
url=https://pietercolpaert.be
]

\cortext[1]{Corresponding author.}

\address[1]{IDLab,
Department of Electronics and Information Systems, Ghent University – imec}

%%
%% Keywords. The author(s) should pick words that accurately describe
%% the work being presented. Separate the keywords with commas.
\begin{keywords}
  Linked data \sep
  Link Traversal Query Processing \sep
  Fragmented database \sep
  Descentralized environments
\end{keywords}

%%
%% This command processes the author and affiliation and title
%% information and builds the first part of the formatted document.
\maketitle

\begin{abstract}
% <!-- Context -->
Link Traversal queries on the web face challenges in source selections and completeness definition due to the size of the web.
Reachability criteria provide completeness conditions based on the traversal of links in the internal data store of query engine.
However, the size of the search domain remains the bottleneck of the approach.
% <!-- need -->
Web environments can have structures exploitable by the query engine to perform more precise source selection.
In current practices, reachability criteria are also used for source selection.
The criteria relies on using elements of the query and predefined regular expressions.
Thus, it is difficult to apply these criteria to traverse environments where logical expressions indicate the location of resources.
For example, data produced after the first of September is stored at \texttt{ex:afterFirstSeptember}.
% <!-- task -->
We propose a rule-based reachability criteria for source selection by capturing logical expression expressed in hypermedia descriptions.
%<!-- object -->
In this poster paper, we show how the Comunica Web querying engine is modified to, during link traversal, take hints from the TREE hypermedia controls, to more efficiently query over subsets of a sensor dataset published using the TREE hypermedia specification.
% <!-- findings -->
Our preliminary findings show that by using this strategy, the query engine can significantly reduce the number of HTTP requests and the query execution time when executing queries aligned with the hypermedia controls.
% <!-- conclusion -->
Given the promising result of this initial approach, we are going to extend our solver beyond a time-based criterion, to support general reasoning over linked data.
\end{abstract}

\section{Introduction}

\sepfootnotecontent{sf:treeSpec}{
    \href{https://treecg.github.io/specification/}{https://treecg.github.io/specification/}
}


\href{https://lod-cloud.net/#diagram}{The increasing amount of available Linked Data on the Web} prompts the need for efficient query interfaces.
During a typical SPARQL query execution, the interface takes the whole query load and delivers the results to the client.
This may cause high workloads which are partially the reason for the historically low availability of SPARQL endpoints~\cite{aranda2013}.
Researchers and practitioners have made efforts to introduce alternative Linked Data publication methods that enable client's participation in the query execution process~\cite{Verborgh2016TriplePF}.
The goal is to lower server-side workloads while keeping fast query execution to the client~\cite{Azzam2021}.
The TREE hypermedia specification is an effort in that direction~\cite{ColpaertMaterializedTREE, lancker2021LDS}, that introduces the concept of domain-oriented fragmentation of large RDF datasets.
For example, in the case of periodic measurements of sensor data, a fragmentation can be made based on the publication date of each data entity.
TREE aims describe datasets fragmentation in ways that enables clients to easily fetch query-relevant subsets.
The data inside a fragment are bounded with constraints that are expressed using hypermedia descriptions~\cite{thomasFieldingPhdThesis}.
More precisely, each fragment describes the constraints of the data of any reachable fragment.
Because of the hyperlinked nature of the documents network, clients must traverse them to find the relevant data to answer their queries.
We propose to use Link Traversal Query Processing (LTQP)~\cite{Hartig2016} as a query mechanism to perform those queries.

LTQP starts by dereferencing a set of URLs called seed URLs~\cite{Hartig2016}.
From these dereferenced documents, links to other documents are dereferenced recursively and their data is inserted into the internal data store of the engine.
LDQL~\cite{hartig2016Ldql} is a theoretical query language to define the traversal of LTQP queries.
However, LDQL is centered around nested regular expressions, thus, it is difficult to define the traversal of links based on domain-specific logical expressions.
The subweb specifications~\cite{bogaerts_rulemlrr_2021}, propose allowing the data provider to define traversal paths with respect to the information related to them.
Thus, given that the query engine trusts the data publisher it can adapt its traversal to reduce its execution time by reducing the number of HTTP requests to get the desired information.
Akin to LDQL, it is difficult with the subweb specifications to express domain-specific logical expressions.
\jr{Can you provide a small example in natural language of what it means to follow links based on domain-specific logical expressions? 
I imagine something like follow only links that contain data produced after X time or data located in Y location}

In this paper, we propose to use an arithmetic boolean solver as the main source selection mechanism for a reachability criterion to traverse TREE documents.
The operators are defined by the \href{https://treecg.github.io/specification/}{TREE specification}~\sepfootnote{sf:treeSpec}.
As a concrete use case for the paper, we consider the publication of (historical) sensor data.
An example query is presented in Figure~\ref{lst:system} with the triples representing the link between two fragments.
The dataset was produced from the DAHCC~\cite{dahcc_resource} dataset and adapted to the TREE specification.

\begin{figure}[h]
    \begin{minipage}{0.50\textwidth}
        \centering
        \lstinputlisting[language=,frame=single]{code/example_sparql_query.ttl}
    \end{minipage}
    \hspace{0.05\textwidth}
    \begin{minipage}{0.43\textwidth}
        \centering
        \lstinputlisting[language=,frame=single]{code/example_tree_relation.ttl}
    \end{minipage}
    \caption{On the left, is a SPARQL query to get sensor measurements and metadata of a fragmented dataset.
    On the right, is the hypermedia description of the location and constraint of the next fragment.
    The constraint describes publication times ($?t$) where $?t>= \text{2022-01-03T09:47:59.000000}$.}
        \label{lst:system}
\end{figure}
\section{A rule-based reachability criterion}

\sepfootnotecontent{sf:opensSource}{
The implementation, the queries and the benchmark are available at the following links:\newline
\href{https://github.com/constraintAutomaton/comunica-feature-link-traversal/tree/feature/time-filtering-tree-sparqlee-implementation}{https://github.com/constraintAutomaton/comunica-feature-link-traversal/tree/feature/time-filtering-tree-sparqlee-implementation}\newline  
\href{https://github.com/constraintAutomaton/How-TREE-hypermedia-can-speed-up-Link-Traversal-based-Query-Processing-queries/tree/main/code}{https://github.com/constraintAutomaton/How-TREE-hypermedia-can-speed-up-Link-Traversal-based-Query-Processing-queries/tree/main/code}\newline
\href{https://github.com/TREEcg/TREE-Guided-Link-Traversal-Query-Processing-Evaluation/tree/main}{https://github.com/TREEcg/TREE-Guided-Link-Traversal-Query-Processing-Evaluation/tree/main}
}

Most research around LTQP centered around query execution in Linked Open Data environments.
Given the pseudo-infinite number of documents on the Web, traversing over all documents is practically infeasible.
To define completeness, different reachability criteria~\cite{hartig2012} were introduced to allow the discrimination of links.
Recently, an alternative direction was introduced where the query engine uses the structure from the data publisher to guide itself towards relevant data sources~\cite{taelman2023, verborgh2020}.

We define our approach as a rule-based reachability criterion.
Our approach builds upon the concept of guided link traversal, and structural assumptions~\cite{taelman2023} to exploit the structural properties of TREE datasets to prune irrelevant links.
Concretely, we interpret the hypermedia descriptions of constraints in TREE fragments as boolean equations.
Upon discovery of a document, the query engine gathers the relevant triples to form the boolean expression of the constraint on the data of the next fragment.
After the parsing of the expression, the filter expression of the SPARQL query is pushdown into the engine's source selection component.
Lastly, the two boolean expressions are evaluated to determine if they can be satisfied, upon satisfaction the IRI targeting the next fragment is added to the link queue otherwise the IRI is pruned.


\subsection{Experimental results}

We implemented our approach using the LTQP version of the query engine Comunica~\cite{comunica}.
To evaluate it, we executed four queries similar to the one in Figure \ref{lst:system}~\sepfootnote{sf:opensSource}.
They were executed over the DAHCC participant 31 dataset (487 MB) with a timeout of two minutes.
We fragmented the dataset following the TREE specification with a tree topology with 100 and 1000 nodes ($n$).


\begin{table}[ht]
    \centering
    \begin{tabular}{|c|c|c|c|c|c|c|c|}
        \hline
        \textbf{n} & \textbf{Query} & \textbf{Time-predicate (ms)}  & \textbf{Time-rule (ms)} & \textbf{Req-rule} & \textbf{Res-rule} \\
        \hline
        100 & Q1 & x & 8,892& 3 & 0 \\
        100 & Q2 & x & 3,541& 3 & 1 \\
        100 & Q3 & x & 59,274& 8 & 8,166 \\
        \hhline{|=|=|=|=|=|=|=|=|}
        1000 & Q1 & x & 1,171& 3 & 0 \\
        1000 & Q2 & x & 734& 3 & 1 \\
        1000 & Q3 & x & 39,987& 51 & 8,166 \\
        \hline
    \end{tabular}
    \caption{
    The predicate-based reachability criterion is not able execute the queries. 
    The rule-based criterion perform better with a larger number of fragments.
    X means that the query did not finish before the timeout.
    Q4 is not depicted because they were not able to terminate before the timeout.}
    \label{tab:result}
\end{table}

The queries were executed using two configurations.
In the first configuration, we use a predicate-based reachability criterion where the engine follows each link of the fragmented dataset.
For the second one, we use our rule-based reachability criterion approach.
As shown in Table \ref{tab:result} no queries could be answered within the timeout by following every fragment.
A possible explanation is the high number of HTTP requests performed~\cite{Hartig2016} leading to non-relevant data sources to answer the queries.
With our rule-based reachability criterion, we see that the queries performed over the 1000 nodes fragmentation perform better than the ones with 100 nodes, particularly when the query has one or zero results.
In those cases, the query execution time is almost one order of magnitude lower compared to the one with 100 nodes.
With Q3 we see that the percentage of reduction is 32\%, this lowering of performance might be caused by the increase by a factor of 17 in HTTP requests.
This raises an interesting observation because we do not observe a reduction of execution time proportional to the reduction of HTTP requests.
Previous research has proposed that inefficient query plan might be the bottleneck of some queries in structured environments~\cite{taelman2023,eschauzier_quweda_2023}.
However, our results seems to show that the size of the internal data source might have a bigger impact on performance than noted in previous studies.
The query Q4 was not able to be answered, with both fragmentations, because it covers a far larger range of publication date and, hence, requires more data to be downloaded and more processing time.

\input{section/Conclusion}

% --- Bibliography ---

\bibliography{references}

\end{document}

%%
%% End of file
